% magic comments
% !TEX program = xelatex

\documentclass{resume}

\usepackage{hyperref}
\usepackage{fontawesome}
\usepackage{fontspec}
%suppress page number
\usepackage{nopageno}

\setmainfont[
 UprightFont={Ethiopia Jiret},
 BoldFont={Ethiopic Yebse}, 
 ItalicFont={Ethiopia Jiret},
 BoldItalicFont={Ethiopia Jiret}
 ]{Linux Libertine O}
% ----------------------------------------------------------------------------- %

\begin{document}
%\fontspec{Ethiopia Jiret}
%\setmainfont{Ethiopia Jiret}

\name{ሚሊዮን አባይነህ }

\medskip

\basicinfo{
  \faPhone ~ (+39) 380-698-xxxx
  \textperiodcentered\
  \faEnvelope ~ \href{mailto:million.abayneh@gmail.com}{million.abayneh@gmail.com}  
  \textperiodcentered\
  \faGithub ~ \href{https://github.com/millzon}{github/millzon}
  \textperiodcentered\
  \faLinkedin   ~ \href{http://www.linkedin.com/in/millzon}{million} 
}

\section{የስራ ተሞክሮ}

% ---
\datedproject{ፊያት (Fiat Crysler Automobiles)}{የልዩ ባለሙያ መሐንዲስ}{መጋቢት 2011 አ.ም. -- አሁን}
\vspace{0.4ex}

\begin{itemize}
  \item አውቶሞቲቭ CAN መረብ እመልካች ሲግናሎች ላይ የመኪና ችግሮችን መለየት ፣ የክላስተር እና ዲፕ ለርኒንግ  ስልተ ቀመሮችን ማጎልበት:- የዝመኑን ቴክኖሎጂ በማልማት በተሽከርካሪዎች የሙከራ ጉዞዎች ላይ መተግበርና ያልተለመዱ ሁናቴዎች ምልክቶችን መለየት፡፡ R, Python, Keras, Tensorflow
  \item የተለያዩ የትግበራ ዑደት መሳሪያዎች (ALM tools ለምሳሌ Polarion, Jira...) መካከል የውህደት ስርአቶችን ማዘጋጀት።

\end{itemize}

% ---
\datedproject{የቱሪን ፖሊቴክኒክ ዩኒቨርሲቲ}{ተመራማሪ}{ህዳር 2008 አ.ም. -- የካቲት 2011 አ.ም.}
\vspace{0.4ex}

\begin{itemize}
  \item ማሽን ለርኒንግ በመጠቀም በቴሌኮም ኔትወርክ ችግሮች አላርም ስርዓት ላይ ለ አውታረ መረብ ክወና ማዕከል ከዋኞች (NOC operators) አጋዥ የሚሆን መፍትሄ መገንባት።

  \item የቤት ዉስጥ ኤሌክትሪክ አቃዎች ኤሌክትሪክ ፍጆታ መገመትና ማሳወቅ የሚያስችል የማቺን ለርኒንግ ስልተ-ቀመር ማዘጋጀት። Python machine learning tools, SiteWhere IoT 

\end{itemize}
% ---
\datedproject{ሚዶሪ (tech Startup)}{ተሲስ/አንብሮ ተመራማሪ}{መስከረም 2007 አ.ም. -- ሀምሌ 2008 አ.ም.}
\vspace{0.4ex}

\begin{itemize}
  \item የማሽን ለርኒንግ ቴክኒኮችን በመጠቀም የሰው ጣልቃ-ገብ የማያስፈልገው ኤሌክትሪክ ጫና ቁጥጥር (NILM) የሶፍትዌር መሠረተ ልማት መገንባት 

  \item ስማርት ሜትሮች እና የ አይኦቲ IoT የግንኙነት አዘገጃጀት ዲዛይን ማድረግ እና መገንባት  

\end{itemize}

% ---

\datedproject{ሳይበርሶፍት}{የሶፍትዌር መሐንዲስ}{መስከረም 2000 አ.ም. -- መስከረም 2005 አ.ም.}
\vspace{0.4ex}

\begin{itemize}
  \item የኢንተርፕራይዝ ሶፍትዌር ስርዓቶችን ዲዛይን ማድረግ ፣ ማልማት እና መተግበር :- የተጠቃሚ ፍላጎቶች ትንተና ፣
የፕሮግራም ተግባራት ፕሮግራም ማድረግ ፣ ሙከራ ማካሄድ ፣ የኮድ ግምገማዎች እና መተግበሪያዎችን ለተጠቃሚዎች ማሰማራት ፡፡

  \item ፕሮጀክቶች

  \begin{itemize}
  \item Balanced Scorecard Information System (BSC)

  \item Integrated Finance Management Information System (IFMIS)

  \item Integrated Finance Management Information System (IFMIS)

  \item Pensioners Payroll Administration System (PensionAdmin)

  \item Graduate TAX Collection System (Cost Sharing)

\end{itemize}

\end{itemize}

% ---

% ----------------------------------------------------------------------------- %

% ----------------------------------------------------------------------------- %
\section{ትምህርት}
\dateditem{\textbf{የቱሪን ፖሊቴክኒክ ዩኒቨርሲቲ (ቱሪን ፤ ጣልያን)} \quad የኤሌክትሮኒክስ መሐንዲስ \quad ሁለተኛ ዲግሪ} {2005 አ.ም. -- 2008 አ.ም.}
የምረቃ ምርምር/አንብሮ ፦ በ ዘመናዊ ወይም ስማርት ከተሞች ውስጥ የኤሌክትሪክ ኃይል ፍጆታ ቁጥጥር ሥርዓቶች የሶፍትዌር መሠረተ ልማት ዲዛይን ማውጣት እና ማጎልበት።

\dateditem{\textbf{አርባ ምንጭ ዩኒቨርሲቲ (አርባምንጭ ፤ ኢትዮጵያ)} \quad ኤሌክትሪክ መሐንዲስ \quad የመጀመርያ ዲግሪ } {1996 አ.ም.-- 1999 አ.ም.}
የምረቃ ምርምር ፦ በ አካባቢያዊ አውታረመረብ ውስጥ የመልቲሚዲያ ኮንፈረንስ ስርዓት ማበጀት።  

% ----------------------------------------------------------------------------- %



\section{የሙያ ችሎታ}
\smallskip
\textbf{ፕሮግራም}: C\#, Python, C/C++ \\
\textbf{የመረጃ አደረጃጀት}: MS SQL Server, Oracle, NoSQL \\
\textbf{ድህረገጽ}: ASP.NET MVC, .NET Core, HTML, CSS, Bootstrap, Blazor \\
\textbf{ሌሎች}: Machine Learning, UML, Scrum, Docker, Microsoft Azure \\
\textbf{የግንኙነት ክህሎቶች}: ጥሩ የመግባባት ችሎታ ፣ የማወቅ ፍላጎት ፣ በራስ ተነሳሽነት ፣ ሥራን በሰዓቱ ማድረስ እና ኃላፊነትን መውሰድ የሚችል \\
\textbf{ቋንቋዎች}: አማርኛ ፣ እንግሊዝኛ ፣ ጣልያንኛ

% ----------------------------------------------------------------------------- %

\section{የምስክር ወረቀት}
\datedcertificate{Scrum Master}{\small{Scrum Master Accredited Certification™}} {2009 አ.ም.}
\datedcertificate{MCITP: Database Administrator}{\small{Microsoft Certified IT Professional}}{2004 አ.ም.}
\datedcertificate{MCPD: ASP.NET Developer}{\small{Microsoft Certified Professional Developer}}{2003 አ.ም.}
\datedcertificate{.NET Framework}{\small{Microsoft Certified Technology Specialist}}{2001 አ.ም.}
\medskip

% ----------------------------------------------------------------------------- %
\section{ህትመት}
\datedpublication{Supporting Telecommunication Alarm Management System with Trouble Ticket Prediction}{IEEE-Transactions on Industrial Informatics} {2012 አ.ም.}
\datedpublication{A Cloud-based On-line Disaggregation Algorithm for Home Appliance Loads}{IEEE-Transactions on SmartGrid™} {2010 አ.ም.}
\datedpublication{An IoT Realization in an Interdepartmental Real Time Simulation Lab for Distribution System Control
and Management Studies}{16th International Conference on Environment and Electrical Engineering} {2008 አ.ም.}
\medskip

% ----------------------------------------------------------------------------- %

\end{document}
